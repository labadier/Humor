\documentclass[a4paper,11pt,twocolumn,twoside]{article}
\usepackage[dvips]{graphicx}
\usepackage{sepln_en}
\usepackage{fullname_esp}
\usepackage[utf8]{inputenc}
\usepackage[spanish,es-nosectiondot, es-tabla, es-noindentfirst, es-nolists]{babel}


\setlength\titlebox{6.6in} %esto por defecto

\title{Poor Robustness or Humor Degeneration ?}

\author {\textbf{Roberto Labadie Tamayo$^1$} \textbf{Mariano Rodriguez Cisneros$^2$} \\
 \textbf{Reynier Ortega  Bueno$^1$}  \textbf{Paolo Rosso$^1$}\\
$^1$Universitat Politècnica de València\\
$^2$Universidad de Oriente, Cuba\\
rlabtam@posgrado.upv.es, mjasonrc@gmail.com, rortega@prhlt.upv.es\\
prosso@dsic.upv.es\\
}

\seplntranstitle{¿Poca robustez o degeneración del humor?}

\seplnclave{Palabras, palabras, palabras en castellano...}

\seplnresumen{Resumen del artículo en castellano con una sangría a izquierda y
derecha de 1 cm, justificado por ambos lados, con tamaño de fuente
11.Resumen del artículo en castellano con una sangría a izquierda y
derecha de 1 cm, justificado por ambos lados, con tamaño de fuente
11.Resumen del artículo en castellano con una sangría a izquierda y
derecha de 1 cm, justificado por ambos lados, con tamaño de fuente
11.Resumen del artículo en castellano con una sangría a izquierda y
derecha de 1 cm, justificado por ambos lados, con tamaño de fuente
11.Resumen del artículo en castellano con una sangría a izquierda y
derecha de 1 cm, justificado por ambos lados, con tamaño de fuente
11.derecha de 1 cm, justificado por ambos lados, con tamaño de fuente
11.derecha de 1 cm, justificado por ambos lados, con tamaño de fuente
11.}


\seplnkey{Palabras, palabras, palabras en inglés...}

\seplnabstract{Resumen del artículo en castellano con una sangría a izquierda y
	derecha de 1 cm, justificado por ambos lados, con tamaño de fuente
	11.Resumen del artículo en castellano con una sangría a izquierda y
	derecha de 1 cm, justificado por ambos lados, con tamaño de fuente
	11.Resumen del artículo en castellano con una sangría a izquierda y
	derecha de 1 cm, justificado por ambos lados, con tamaño de fuente
	11.Resumen del artículo en castellano con una sangría a izquierda y
	derecha de 1 cm, justificado por ambos lados, con tamaño de fuente
	11.Resumen del artículo en castellano con una sangría a izquierda y
	derecha de 1 cm, justificado por ambos lados, con tamaño de fuente
	11.}

\firstpageno{1}


\begin{document}

% la siguiente instrucción sólo se debe usar si el abstract sobrescribe el texto
% la longitud variará según se necesite

%\setlength\titlebox{20cm} % se aumenta el tamaño del espacio reservado para datos de título


\label{firstpage} \maketitle


\section{Introduction}

Proper comprehension of humor goes beyond the semantics involved in utterances. It relies on information from the context where jokes are made in and the background knowledge of the receptor, mainly socio-cultural knowledge. 
\\
The latter implies a different or even null perception from one person to another. Nevertheless, when it comes to figurative devices, such as humor, syntactic characteristics of languages play a critical role in this perspective's loss. Particularly, when information flows from one language to another on its way to the receptor, the intended meaning of a joke is on risk of vanishing.
\\
Word plays are one of those examples of language dependent expressions that can be potentially lost upon translation into a different language, since they employ the arrangement and phonetic of words to produce humor.
For example, in:\\

\textit{Why do male ants float while female ants sink? They're buoy-ant}
\\\\
There is no way to translate the phrases to ensure the understanding by a spanish speaker, regardless their background knowledge.
\\
Neural Models have place new in\textit{ states of the art} in many task related in humor detection.
\section{related Works}
Hola


\section{Methodology}

Como

\section{Experimental Results}

es tu ?

\section{Discussion and Future Work}


\bibliographystyle{fullname}


\end{document}
